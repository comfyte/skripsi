\chapter{Hasil dan Pembahasan}

\section{Analisis pada Sistem Operasi Linux Sungguhan}

Dengan bantuan perangkat lunak virtualisasi UTM, analisis sistem penanganan notifikasi dan sisten kontrol media pada sistem operasi Linux sungguhan dilakukan pada sebuah \textit{virtual machine} (bertindak sebagai komputer referensi) yang menjalankan sistem operasi Ubuntu 22.04 edisi arsitektur ARM 64-bit dan berjalan di atas perangkat komputer \textit{host} MacBook Air keluaran tahun 2020 (berprosesor M1) bersistem operasi macOS 13 "Ventura".



\subsection{Analisis Sistem Penanganan Notifikasi}

\subsection{Analisis Sistem Kontrol Media}

\section{Analisis Ekosistem pada Masing-Masing Platform (Linux dan Windows)}

\subsection{Analisis Sistem Penanganan Notifikasi}

\subsection{Analisis Sistem Kontrol Media}

\section{Pengembangan Perangkat Lunak FancyWSL}

\section{Perbandingan Hasil Penelitian dengan Hasil Terdahulu}
