\textit{
Windows Subsystem for Linux (WSL) was released by Microsoft to support running Linux applications and software directly within the Windows operating system. WSL was introduced shortly after the introduction of Windows 10; since Windows 11, WSL gained a new support for running graphical applications directly within the Windows user interface. The implementation of these still lacks several aspects, such as notification handling support and media control support. This thesis discusses about these two aspects, the background, the detailed implementation, and the final testing. In the end, the implementation of these two aspects will hopefully improve the user experience aspect of WSL usage.
}

\noindent\textbf{Keywords} : WSL, Linux, D-Bus, notification, media control
