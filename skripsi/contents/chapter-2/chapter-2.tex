\chapter{Tinjauan Pustaka dan Dasar Teori}

\section{Tinjauan Pustaka}

Lorem ipsum some placeholder text here.


\section{Dasar Teori}

\subsection{Antarmuka Sistem Operasi (\textit{Shell})}

\subsection{Linux, \textit{Unix-like}, dan Standar POSIX}

\subsection{Dukungan Kompatibilitas Linux atau \textit{Unix-like} di Windows Pra-WSL}

Teknologi yang digunakan dalam skripsi ini, Windows Subsystem for Linux, bukan merupakan satu-satunya usaha yang digunakan untuk membawa kompatibilitas Linux atau Unix-like ke dalam lingkungan sistem operasi Windows. Microsoft memiliki sejarah panjang dalam membawa dukungan Linux atau Unix-like ke dalam lingkungan Windows.

Sebelum mengembangkan WSL, Microsoft memiliki sejumlah usaha terdahulu yang bertujuan membawa dukungan Linux/Unix-like ke dalam sistem operasi Windows. Dalam proses pengembangan Windows NT yang saat ini mendasari versi-versi Windows modern (Windows 2000 dan setelahnya), Microsoft mendesain sistem operasi tersebut sedemikian rupa agar bersifat modular dan mendukung berbagai macam "subsistem". Pada awal pengembangannya, Windows NT direncanakan memiliki tiga buah subsistem, yakni subsistem Win32, subsistem OS/2 (untuk mendukung interoperabilitas dengan sistem operasi OS/2 milik IBM), dan subsistem POSIX (untuk mendukung interoperabilitas dengan sistem-sistem berbasis Linux, UNIX, atau Unix-like yang sedang populer di kalangan bisnis pada masa itu). Seiring waktu, Microsoft memutus dukungan dan pengembangan lebih lanjut terhadap subsistem OS/2 karena adanya permasalahan bisnis dengan IBM.

Berbeda dengan subsistem OS/2 yang telah mati, dukungan dan pengembangan terhadap subsistem POSIX masih terus berjalan. Secara umum, terdapat tiga iterasi teknologi subsistem POSIX/UNIX yang dikembangkan oleh Microsoft:

\begin{enumerate}
    \item \textbf{Microsoft POSIX Subsystem.} Teknologi ini merupakan teknologi asli kembangan Microsoft yang ada sejak awal pengembangan Windows NT. Teknologi ini tersedia untuk sistem operasi Windows NT 3.1 hingga Windows 2000.
    \item \textbf{Windows Services for Unix (SFU).} Produk ini sesungguhnya adalah teknologi milik perusahaan bernama PTS yang dilisensikan kepada Microsoft. Produk ini menggantikan Microsoft POSIX Subsystem sejak sistem operasi Windows XP dan Windows Server 2003.
    %Teknologi yang membawahi (...) ini sejatinya merupakan hasil pengembangan
    \item \textbf{Subsystem for Unix-based Applications (SUA).}
    % Teknologi ini sejatinya merupakan pengembangan lebih lanjut dari SFU.
    % Teknologi ini pada awalnya bernama subsistem Interix dan berganti nama setelah Microsoft mengakuisisi perusahaan pengembangnya. Setelah akuisisi, produk ini tidak lagi berdiri sebagai produk terpisah, tetapi menjadi salah satu komponen dari Windows Services for Unix (SFU). Komponen ini tersedia sejak 
\end{enumerate}

% Reason against elaborationg further: The lack of new sources due to the nature of this topic that is "historical information".

% https://link.springer.com/chapter/10.1007/978-1-4842-6038-8_1

Di samping solusi-solusi yang [telah] ditawarkan oleh Microsoft tersebut, terdapat pula beragam solusi-solusi pihak ketiga yang meraih tujuan yang sama. Solusi-solusi pihak ketiga ini ada baik guna mengisi kekosongan dukungan oleh Microsoft secara langsung maupun sebagai substitusi (\textit{substitute}) yang menggantikan solusi bawaan milik/buatan Microsoft karena [mungkin] solusi tersebut dirasa kurang atau masih belum sempurna.

\begin{enumerate}
    \item \textbf{Colinux.}

    \item \textbf{Cygwin.}
\end{enumerate}

\subsection{Windows Subsystem for Linux (WSL)}

Windows Subsystem for Linux merupakan usaha terbaru Microsoft dalam membawa kompatibilitas Unix/Linux ke dalam lingkungan Windows.

\subsection{Dukungan Kompatibilitas Linux atau \textit{Unix-like} di Windows oleh Pihak Ketiga}

Microsoft tidak sendirian dalam mengembangkan dukungan lingkungan Linux atau \textit{Unix-like} ke dalam sistem operasi Windows; terdapat beragam pilihan pihak ketiga yang dapat membawa dukungan lingkungan Linux atau \textit{Unix-like} ke Windows.

\subsection{Teknologi Penampilan Antarmuka Grafis dan Sistem Perjendelaan (\textit{Windowing System}) pada Lingkungan Linux: X11/Xorg dan Wayland}

\subsection{Dukungan Antarmuka Grafis pada Windows Subsystem for Linux: Windows Subsystem for Linux GUI (WSLg)}

Microsoft pertama kali menyatakan dukungan penjalanan perangkat lunak berantarmuka grafis (GUI) di Windows Subsystem for Linux versi 2 (WSL2) secara bawaan (\textit{built-in}) pada perhelatan pengembang tahunan Microsoft Build 2021.

Dalam pengembangan fungsionalitas antarmuka grafis pada Windows Subsystem for Linux, Microsoft berkomitmen untuk mengikuti standar yang telah ada pada lingkungan Linux.

\begin{figure}
    \centering
    \includegraphics[width=0.5\linewidth]{wslg-architecture.png}
    \caption{Arsitektur Windows Subsystem for Linux GUI (WSLg)}
    \label{fig:enter-label}
\end{figure}

% https://link.springer.com/chapter/10.1007/978-1-4842-6873-5_1
% https://devblogs.microsoft.com/commandline/wslg-architecture/

\subsection{D-Bus}

\subsection{Windows App SDK}

\subsection{\textit{Media Player Remote Interfacing Specification} (MPRIS)}

\subsection{Windows API: System Media Transport Controls (SMTC)}

\subsection{\textit{Standard Stream} dan \textit{Piping}}

\section{Analisis Perbandingan Metode}

Pengerjaan serta pencapaian tujuan skripsi ini, terutama mengenai penghubungan (\textit{bridging}) lingkungan Windows Subsystem for Linux (WSL) dengan lingkungan \textit{host} Windows, dapat ditempuh melalui berbagai macam metode. Setelah meninjau berbagai macam pustaka yang telah ada serta melakukan riset tentang teknologi terkait, penulis merumuskan enam macam pendekatan yang mungkin dilakukan:
\begin{enumerate}
    \item Lingkungan WSL dihubungkan dengan lingkungan \textit{host} Windows melalui \textit{socket} Unix. Hal ini dimungkinkan dengan diperkenalkannya dukungan \textit{socket} Unix (AF\_UNIX) pada sistem operasi Windows 10 ke atas pada tahun 2017 \cite{bringing-afunix-to-windows}. Penelusuran lebih lanjut mengindikasikan bahwa kemampuan ini rupanya hanya mendukung Windows Subsystem for Linux versi 1 (WSL1) dan belum mendukung Windows Subsystem for Linux versi 2 (WSL2) \cite{github-issues-afunix-not-supported-in-wsl2}. Oleh karena itu, mengingat kemampuan grafis (\textit{graphical user interface}) pada Windows Subsystem for Linux hanya tersedia pada Windows Subsytem for Linux versi 2 (WSL2), metode ini tidak relevan dengan tujuan skripsi ini.
    
    \item Lingkungan Windows Subsystem for Linux (WSL) dihubungkan dengan lingkungan \textit{host} Windows dengan mengutilisasikan \textit{named pipes}.
    
    \item Lingkungan Windows Subsystem for Linux (WSL) dihubungkan dengan lingkungan \textit{host} Windows dengan memanfaatkan \textit{server} HTTP. Metode ini ikut dipertimbangkan mengingat pengalaman penulis yang cukup banyak berhubungan dengan bidang pengembangan web (\textit{web development}). Metode ini dapat dibagi kembali menjadi dua submetode:
    \begin{enumerate}
        \item Penghubungan kedua lingkungan melibatkan dua buah \textit{server} HTTP yang masing-masing berjalan di sisi Windows Subsystem for Linux (WSL) dan di sisi \textit{host} Windows. Dua buah \textit{server} diperlukan karena komunikasi bersifat dua arah: WSL mengirimkan konten yang ingin ditampilkan ke sisi \textit{host} Windows dan \textit{host} Windows mengomunikasikan hasil interaksi pengguna (\textit{user input}) kembali ke sisi Windows Subsystem for Linux (WSL). Meskipun implementasi submetode ini dapat dibuat seefisien mungkin, penjalanan dua buah \textit{server} tetap saja terasa kurang efisien, terutama bila dibandingkan dengan metode-metode lainnya.
        \item Penghubungan kedua lingkungan menggunakan cukup satu buah \textit{server} saja untuk menghindari duplikasi penggunaan \textit{resources}, tetapi memanfaatkan teknologi yang memungkinkan komunikasi secara dua arah seperti HTTP \textit{long polling} dan WebSocket. Penggunaan HTTP \textit{long polling} memiliki kemungkinan menghasilkan performa yang kurang efisien \cite{problems-in-http-long-polling}, sedangkan penggunaan WebSocket sama saja dengan penggunaan Unix \textit{socket} biasa namun dengan \textit{overhead} protokol HTTP yang dapat berefek pada performa.
    \end{enumerate}
    
    \item Lingkungan WSL dihubungkan dengan lingkungan \textit{host} Windows dengan komunikasi secara tekstual atau terserialisasi (\textit{serialized}). Pertukaran informasi berbentuk teks (tekstual) ini dapat melalui berbagai perantara; berikut beberapa di antaranya.
    \begin{enumerate}
        \item Informasi tekstual dipertukarkan melalui berkas \textit{executable} pembantu (\textit{helper}) sebagai argumen pengeksekusian berkas \textit{executable} tersebut. Ditentukan dua buah berkas \textit{executable} yang masing-masing bertugas mengirimkan informasi ke sisi \textit{host} Windows dan mengirimkan informasi ke sisi WSL. Pada sisi WSL, hal ini dimungkinkan oleh kemampuan WSL menjalankan berkas \textit{executable} Windows (umumnya berekstensi \verb|.exe|) secara langsung di dalam \textit{command-line} WSL \cite{msdocs-run-windows-tools-from-linux}. Pada sisi Windows, hal ini dimungkinkan oleh kemampuan memanggil WSL secara programatik atau \textit{scripted} dengan perintah yang telah ditetapkan. Sebagai contoh, pengiriman informasi dari sisi WSL dapat dilakukan dengan perintah
        \begin{lstlisting}[language=bash]
# Di shell bash
/path/to/fwsl-send-to-windows.exe --data='<JSON-serialized data>'
        \end{lstlisting}
        dan pengiriman informasi dari sisi Windows ke sisi WSL dapat dilakukan dengan perintah
        \begin{lstlisting}
# Di shell PowerShell
wsl.exe /path/to/fwsl-send-to-wsl --data='<JSON-serialized data>'
        \end{lstlisting}

        \item Informasi tekstual dipertukarkan melalui \textit{named pipes}. Cara pertukaran data ini serupa dengan penggunaan berkas \textit{executable} pembantu (\textit{helper}) pada submetode (a), namun data disalurkan melalui \textit{standard stream} (seperti \textit{standard input} dan \textit{standard output}) dan dihubungkan dengan \textit{named pipes} alih-alih diletakkan sebagai argumen suatu berkas \textit{executable}. Pertukaran data dengan cara ini tetap memerlukan berkas-berkas \textit{executable} sebagai pengirim dan/atau penerima data. Selain berkas-berkas \textit{executable} yang bersangkutan, diperlukan pula Dalam submetode ini, dimungkinkan 
    \end{enumerate}
    
    \item Lingkungan Windows Subsystem for Linux (WSL) dihubungkan dengan lingkungan \textit{host} Windows melalui perangkat lunak tambahan "D-Bus for Windows" yang diinstal di lingkungan \textit{host} Windows.
    
    \item Lingkungan Windows Subsystem for Linux (WSL) dihubungkan dengan lingkungan \textit{host} Windows dengan metode komunikasi tekstual melalui penjalanan dan pembacaan perintah `dbus-send` dan `dbus-monitor`.
    
    \item Lingkungan Windows Subsystem for Linux (WSL) dihubungkan dengan lingkungan \textit{host} Windows dengan metode komunikasi tekstual melalui berkas biner \textit{executable}.
\end{enumerate}

Setelah melalui berbagai pertimbangan di atas, metode keenam dirasa paling cocok diterapkan dalam pengerjaan skripsi ini.

\section{Pertanyaan Tugas Akhir}

Dalam serangkaian 
