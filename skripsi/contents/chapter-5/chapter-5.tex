\chapter{Kesimpulan dan Saran}

\section{Kesimpulan}

Berdasarkan sejumlah penelitian dan pengujian yang telah dilakukan dalam tugas akhir ini, terbukti bahwa penghubungan sejumlah aspek-aspek antarmuka di Windows Subsystem for Linux (WSL) dengan Windows mungkin dilakukan. Melalui tugas akhir ini, dapat diketahui bahwa sistem D-Bus yang aslinya cukup eksklusif digunakan di dalam sistem operasi Linux pada akhirnya dapat diakses pula oleh sistem operasi Windows melalui salah satu protokol komunikasi yang didukung oleh \textit{daemon} D-Bus, yaitu TCP. Tugas akhir ini juga menunjukkan bahwa kekurangan-kekurangan implementasi secara pihak pertama yang dilakukan oleh Microsoft dapat ditambal oleh solusi-solusi pihak ketiga seperti perangkat lunak FancyWSL yang dikembangkan dalam tugas akhir ini.

\section{Saran}

Dalam pengerjaan tugas akhir ini, terutama pada tahap pengembangan perangkat lunak FancyWSL, masih terdapat celah-celah implementasi yang belum diimplementasikan karena berbagai alasan, salah satunya adalah keterbatasan waktu pengerjaan. Oleh karena itu, akan lebih baik apabila perangkat lunak FancyWSL dikembangkan secara lebih baik dan dengan alur pengembangan yang lebih jelas.
