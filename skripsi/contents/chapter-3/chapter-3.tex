\chapter{Metode Penelitian}

\section{Alat dan Bahan Tugas Akhir}

\subsection{Alat Tugas Akhir}

\begin{enumerate}
    \item Perangkat komputer, baik \textit{desktop} maupun komputer jinjing (\textit{laptop}), sebagai \textbf{komputer utama} dengan
    \begin{itemize}
        \item sistem operasi Windows 11 Home, Pro, atau SKU lain yang mendukung kapabilitas Windows Subsystem for Linux versi 2 (WSL2);
        \item arsitektur x86\_64 (64-bit) dengan prosesor seri Intel Core generasi ke-8 atau lebih tinggi;
        \item memori (RAM) minimal 8 GB; dan
        \item ruang penyimpanan minimal 64 GB.
    \end{itemize} Bila menggunakan \textit{virtual machine}, harap pastikan fungsionalitas "\textit{nested virtualization}" bekerja dengan baik karena Windows Subsystem for Linux versi kedua (WSL2) membutuhkan dukungan virtualisasi. Pada tugas akhir ini, digunakan perangkat komputer \textit{all-in-one} ASUS ZN270IE beridentitas "COM25" di Laboratorium Jaringan Komputer dan Aplikasi Terdistribusi, Departemen Teknik Elektro dan Teknologi Informasi, Universitas Gadjah Mada dengan detail sebagai berikut.
    \begin{itemize}
        \item Sistem operasi Windows 11 Pro (versi 22H2) 64-bit
        \item Prosesor Intel Core i7-7700T dengan kecepatan 2,90 GHz dan arsitektur x86\_64 (64-bit)
        \item Memori (RAM) sebanyak 16 GB
        \item Kemampuan layar sentuh dengan multisentuh hingga sepuluh jari secara bersamaan
    \end{itemize}

    \item Perangkat lunak "Windows Subsystem for Linux 2 (WSL2)" versi perilisan terbaru yang terinstal pada komputer utama. Untuk menginstal atau memastikan bahwa WSL sudah versi terbaru, jalankan perintah berikut pada PowerShell atau Command Prompt.
    \begin{lstlisting}
wsl --install\end{lstlisting}
    Apabila terdapat perangkat lunak \textit{virtual machine} pihak ketiga (VirtualBox, VMware, dan sebagainya) yang terinstal pada komputer utama tersebut, harap pastikan bahwa perangkat lunak \textit{virtual machine} tersebut telah diperbarui ke versi paling baru dan kompatibel dengan WSL2 atau Hyper-V (kapabilitas yang menenagai WSL2); instalasi WSL2 dan perangkat lunak \textit{virtual machine} pihak ketiga tersebut dalam sistem yang sama dapat menyebabkan degradasi performa pada perangkat lunak \textit{virtual machine} pihak ketiga tersebut karena perangkat lunak \textit{virtual machine} pihak ketiga tersebut mencoba mengakomodasi aktifnya kapabilitas Hyper-V.


    \item \textbf{Opsional:} Perangkat komputer bersistem operasi Ubuntu 22.04 (LTS) sebagai \textbf{komputer referensi (\textit{reference machine})}. Perangkat komputer dapat berupa komputer \textit{desktop} ataupun komputer jinjing (\textit{laptop}). Apabila tidak memungkinkan, mesin referensi dapat disubstitusikan dengan \textit{virtual machine} yang berjalan di suatu perangkat komputer \textit{host} melalui perangkat lunak \textit{virtual machine}. Pada tugas akhir ini, penulis menggunakan \textit{virtual machine} yang menjalankan sistem operasi Ubuntu 22.04 (LTS) melalui perangkat lunak virtualisasi "UTM" dengan \textit{host} komputer jinjing Apple MacBook Air keluaran tahun 2020 (berprosesor M1) bersistem operasi macOS 13 "Ventura".
    
    
    \item Perangkat lunak untuk diujikan: Microsoft Edge for Linux, (...), dan (...). Semua perangkat lunak ini akan diinstal di dalam lingkungan Windows Subsystem for Linux versi kedua (WSL2).

    \item Perangkat lunak "Visual Studio 2022" edisi Community guna mengembangkan perangkat lunak penjembatan (\textit{bridge}) di sisi Windows.

    \item Perangkat lunak editor teks atau editor kode untuk keperluan pengembangan perangkat lunak yang tidak menggunakan Visual Studio, seperti pengimplementasian perangkat lunak di sisi WSL. Pada tugas akhir ini, digunakan "Visual Studio Code" sebagai editor kode.

    \item Perangkat lunak \textit{compiler} yang memadai yang terinstal di lingkungan WSL. (TODO: Decide what language to use!)
\end{enumerate}

\subsection{Bahan Tugas Akhir}

Bahan tugas akhir adalah segala sesuatu yang bersifat fisik atau digital yang digunakan untuk kebutuhan tugas akhir. Bahan tugas akhir dapat berupa:

\begin{enumerate}
	\item Bahan habis pakai. Bahan yang digunakan untuk tugas akhir. Sebagai contoh 
	mungkin dibutuhkan kertas transparansi, baterai, atau yang lain 
	\item Bahan yang berupa data atau informasi yang menjadi dataset tugas akhir. Dataset tugas akhir dapat berupa:
\end{enumerate}
\begin{itemize}
	\item Dataset pihak lain yang diperoleh dengan izin atau dalam lisensi yang diizinkan untuk digunakan secara langsung 
	\item Dataset pihak pertama yang disusun sendiri melalui quisioner, observasi, atau interview 
	\item Dokumen panduan yang mengacu pada standar, hasil tugas akhir, atau artikel yang disitasi dan digunakan.
\end{itemize}

\section{Alur Tugas Akhir}

Lorem ipsum.

\section{Metode yang Digunakan}

Lorem ipsum.

\section{Metode Analisis Data}

Lorem ipsum.
