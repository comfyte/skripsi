\chapter{Pendahuluan}

\section{Latar Belakang}

Penggunaan komputer, terutama dalam bidang teknologi informasi dan pengembangan perangkat lunak (\textit{software development}), dipenuhi oleh pilihan sistem operasi yang beragam. Menurut survei pengembang tahunan yang diselenggarakan oleh Stack Overflow, sistem operasi Windows menempati urutan pertama sistem operasi yang paling banyak digunakan oleh pengembang perangkat lunak pada tahun 2022 \cite{stackoverflow-developer-survey-2022-most-popular-os} dan 2023 \cite{stackoverflow-developer-survey-2023-most-popular-os}, baik dalam konteks penggunaan personal maupun dalam konteks penggunaan profesional. Sistem operasi Linux menempati posisi kedua setelah Windows pada pemeringkatan yang sama yang menunjukkan bahwa Linux juga cukup diminati oleh orang-orang yang bergerak di bidang teknologi informasi. Berikut beberapa di antara faktor-faktor yang berkontribusi pada hal ini.
\begin{itemize}
    \item Linux memiliki sifat-sifat dan karakteristik yang sejalan dengan para pengembang. Linux, baik \textit{kernel}-nya maupun komponen-komponen lainnya, bersifat sumber terbuka (\textit{open source}); pengguna yang memiliki keterampilan dapat memodifikasi aspek-aspek Linux yang diinginkan. Linux juga bersifat modular, sehingga siapa pun dapat menukar-nukar komponen-komponen penyusun Linux sesuai keinginan atau kebutuhan, mulai dari jenis \textit{kernel}, proses/mekanisme pemulaian sistem operasi seperti \textit{bootloader}, metode pengelolaan paket perangkat lunak (\textit{software packages}), hingga penampilan grafis (GUI) seperti pemilihan lingkungan \textit{desktop} (\textit{desktop environment}) yang dirasa paling cocok dengan pengguna.

    \item Linux memiliki banyak varian distribusi yang dapat dipilih oleh pengguna. Sebuah distribusi Linux sejatinya adalah satu set komponen-komponen penyusun Linux yang disajikan sebagai satu kesatuan paket sistem operasi; karena Linux bersifat modular, terdapat banyak kombinasi komponen-komponen yang bisa dipilih untuk dijadikan suatu distribusi Linux. Beberapa contoh distribusi Linux terpopuler yaitu Ubuntu, Linux Mint, Fedora, dan Arch Linux.

    \item Linux memiliki dukungan peralatan pengembangan yang sangat luas. Secara \textit{default}, sejumlah distribusi Linux telah dilengkapi oleh alat-alat pengembangan yang sering digunakan oleh pengembang, seperti Git, \verb|curl|, GCC (GNU Compiler Collection), dan interpreter bahasa pemrograman Python. Beberapa alat pengembangan seperti \textit{framework} web Jekyll bahkan secara resmi hanya mendukung instalasi pada sistem operasi berbasis "mirip Unix", termasuk Linux dan macOS, dan hanya mendukung instalasi pada Windows secara tidak resmi \cite{jekyll-windows-unofficially-supported}.

    \item Linux masih berada di dalam satu keluarga dengan sistem-sistem operasi "mirip Unix" lain, seperti macOS, FreeBSD, dan Solaris. Hal ini dapat memudahkan interoperabilitas bagi pengguna yang bekerja dengan set teknologi Unix.
\end{itemize}

Di samping itu, terdapat sektor-sektor teknologi yang saat ini didominasi oleh Linux, salah satunya yaitu sektor infrastruktur internet seperti \textit{server} dan \textit{cloud}. Dilansir oleh situs berita teknologi ZDNet pada tahun 2015, data yang diperoleh oleh Alexa (layanan pemeringkatan situs-situs web) menunjukkan bahwa 96,3\% dari satu juta \textit{server} web teratas menggunakan sistem operasi Linux \cite{zdnet-alexa-website-ranking-linux-dominance}. Selain itu, The Linux Foundation melalui laporan Linux Kernel Development pada tahun 2017 menunjukkan bahwa Linux juga mendominasi penggunaan sistem operasi di layanan \textit{public cloud} \cite{linux-dominance-on-public-cloud}. Mengetahui hal ini, para pengembang yang bekerja dalam sektor infrastruktur internet seperti \textit{server} dan \textit{cloud} secara natural akan cenderung memilih Linux sebagai sistem operasi tempat mereka bekerja sehari-hari guna memaksimalkan interoperabilitas. Meskipun demikian, tidak sedikit pula pengembang yang memilih menggunakan sistem operasi Windows untuk bekerja; keadaan ini dapat menimbulkan ketidakserasian mengingat para pengembang tersebut tidak menggunakan sistem operasi yang sama yang digunakan dalam target pengembangan mereka.

Windows Subsystem for Linux (WSL) merupakan teknologi yang memungkinkan pengguna menjalankan perangkat lunak dan peralatan-peralatan pengembangan (\textit{development tools}) berbasis Linux secara langsung di dalam sistem operasi Windows 10 atau Windows 11 tanpa memerlukan pergantian sistem operasi. Didasari oleh banyaknya permintaan pengguna untuk lingkungan pengembangan (\textit{development environment}) di Windows yang lebih baik \cite{some-background-on-why-bouow-was-planned}, Windows Subsystem for Linux pertama kali diperkenalkan oleh Microsoft melalui pembaruan Windows 10 bertajuk "Anniversary Update" (pembaruan Juli 2016) sebagai fitur eksperimental (\textit{beta}) dengan nama "Bash on Ubuntu on Windows" \cite{bouow-release-article}. Nama awal tersebut kemudian dihilangkan dan diubah menjadi cukup "Windows Subsystem for Linux" untuk menandakan dibukanya dukungan distribusi Linux di luar Ubuntu. Teknologi ini terus berkembang seiring waktu dan umpan balik dari pengguna hingga melepas titel "\textit{beta}"-nya dan menjadi fitur stabil secara resmi pada perilisan pembaruan Windows 10 bertajuk "Fall Creators Update" pada September 2019. Pada tahun 2020, seiring dengan peningkatan versi menjadi bernama Windows Subsystem for Linux versi kedua (WSL2), Windows Subsystem for Linux melalui perubahan arsitektur besar yang meningkatkan kompatibilitas terhadap fungsi-fungsi Linux (\textit{system calls}) secara signifikan dengan dibawanya \textit{kernel} Linux sungguhan yang tervirtualisasi alih-alih hanya mengandalkan lapisan translasi (\textit{translation layer}).
% TODO: Cari sitasi/sumber yang pas untuk waktu perilisan WSL pertama di W10 Anniversary Update

Pada perilisan sistem operasi Windows 11, pengembangan Windows Subsystem for Linux memasuki babak baru dengan diresmikannya dukungan penjalanan perangkat lunak berantarmuka grafis (\textit{graphical user interface} atau GUI) melalui fitur bernama "Windows Subsystem for Linux GUI" (WSLg) yang berbasis pada peningkatan yang dilakukan pada WSL2. Hal ini merupakan kemajuan yang signifikan karena kini pengguna tidak lagi terbatas pada perangkat lunak berantarmuka baris perintah (\textit{command line}) saja. Perkembangan ini membuka dukungan yang lebih luas kepada perangkat-perangkat lunak berbasis Linux yang ada, beberapa di antaranya berantarmuka grafis.

Saat ini, Windows Subsystem for Linux telah terintegrasi dengan cukup baik di dalam lingkungan antarmuka Windows secara keseluruhan berkat peningkatan-peningkatan yang telah dilakukan. Pengguna dapat dengan mudah meluncurkan aplikasi atau perangkat lunak berbasis Linux secara langsung di Windows dan perangkat lunak tersebut akan tampil secara harmonis di samping perangkat-perangkat lunak Windows \textit{native} lainnya dan bertindak layaknya perangkat-perangkat lunak \textit{native} tersebut. Sistem perjendelaan (\textit{windowing}) sederhana seperti operasi \textit{maximize}, \textit{restore}, dan \textit{minimize} pada perangkat-perangkat lunak berbasis Linux tersebut bertindak layaknya perangkat lunak Windows \textit{native} pada umumnya sesuai ekspektasi pengguna; dalam sebagian besar kasus penggunaan (\textit{use case}), tingkat integrasi ini sudah cukup dalam membuat perangkat-perangkat lunak berbasis Linux tersebut terasa harmonis. Namun, begitu pengguna mencoba melakukan operasi-operasi yang lebih dari itu, seperti penerimaan notifikasi, \textit{window snapping}, dan kontrol media pada aplikasi-aplikasi musik, pengguna akan mendapati bahwa beberapa operasi tersebut tidak bisa dilakukan karena belum didukung. Dengan kata lain, masih terdapat disparitas pada fungsionalitas sistem perjendelaan antara perangkat-perangkat lunak berbasis Linux yang berjalan di WSL dan perangkat-perangkat lunak berbasis Windows yang berjalan secara \textit{native} dikarenakan belum sempurnanya integrasi Windows Subsystem for Linux GUI ke dalam lingkungan Windows secara keseluruhan.

Melalui tugas akhir ini, penulis mengeksplor potensial-potensial penyempurnaan yang dapat dilakukan untuk meningkatkan situasi yang ada. Penulis berfokus pada peningkatan dalam dua aspek, yaitu aspek penampilan notifikasi dan aspek kontrol media. Namun, dapat dipahami bahwa implementasi sistem Linux di WSL bukan merupakan implementasi yang sempurna dan terdapat celah-celah fungsionalitas yang umumnya ada pada distribusi Linux normal, mengingat WSL tidak didesain sebagai sebuah distribusi Linux utuh, yang dapat menghambat pengimplementasian kedua aspek awal tersebut, sehingga diperlukan pula pengimplementasian hal-hal penunjang yang mungkin diperlukan. Penulis berharap bahwa penyempurnaan yang penulis lakukan dapat bermanfaat bagi pengguna-pengguna sistem operasi Windows yang bergantung pada Windows Subsystem for Linux. Penulis percaya bahwa usaha yang dilakukan dalam penelitian ini memiliki efek jangka panjang meningkatkan pengalaman pengguna (\textit{user experience)} secara keseluruhan. Di samping itu, tidak menutup kemungkinan bahwa penyempurnaan yang penulis lakukan dapat bermanfaat pula bagi tim pengembang \textit{upstream} yang bekerja mengembangkan fitur Windows Subsystem for Linux ini.


\section{Rumusan Masalah}

"Bagaimana menciptakan lingkup integrasi yang lebih menyeluruh dan lebih komprehensif dibandingkan dengan lingkup integrasi yang telah ada, terutama dalam aspek penampilan notifikasi dan pengontrolan media, guna memaksimalkan kepuasan pengguna dalam menggunakan perangkat-perangkat lunak berbasis Linux dan berantarmuka grafis (GUI) di Windows Subsystem for Linux (WSL)?" \textcolor{orange}{(FIXME)}


\section{Tujuan Penelitian}

\begin{enumerate}
    \item Menganalisis sistem penanganan notifikasi dan sistem kontrol media pada sistem operasi Linux sungguhan sebagai basis implementasi di Windows Subsystem for Linux (WSL).

    \item Menganalisis ekosistem penanganan notifikasi dan pengontrolan media di Linux dan Windows serta potensi penghubungan (\textit{bridging}) keduanya.

    \item Mengimplementasikan fungsi penanganan notifikasi dan kontrol media pada Windows Subsystem for Linux (WSL) hingga terintegrasi dengan baik dengan \textit{shell} Windows melalui perangkat lunak FancyWSL yang dikembangkan.
\end{enumerate}


\section{Batasan Penelitian}

\begin{enumerate}
    \item \textbf{Objek penelitian:} Penelitian dilakukan pada instalasi perangkat lunak Windows Subsystem for Linux versi 2.0.9 di sebuah perangkat komputer bersistem operasi Microsoft Windows 11 Pro berarsitektur x86\_64 (64-bit).
    
    \item \textbf{Metode penelitian:} Fungsi-fungsi tambahan yang perlu dikembangkan tidak ditambahkan secara langsung ke kode sumber (\textit{source code}) inti Windows Subsystem for Linux (WSL), tetapi ditambahkan ke perangkat lunak pembantu (\textit{helper}) terpisah yang melengkapi penjalanan perangkat lunak Windows Subsystem for Linux (WSL) utama.
    
    \item \textbf{Waktu dan tempat penelitian:} Penelitian dan percobaan dilakukan dari April 2023 hingga Desember 2023 di perangkat komputer beridentitas "COM25" di Laboratorium Jaringan Komputer dan Aplikasi Terdistribusi, Departemen Teknik Elektro dan Teknologi Informasi, Universitas Gadjah Mada. Untuk memudahkan pengerjaan, dimungkinkan pengaksesan perangkat komputer secara jarak jauh (\textit{remote access}) menggunakan bantuan perangkat lunak Microsoft Remote Desktop.
    
    \item \textbf{Populasi dan sampel:} Pengerjaan dan evaluasi seluruhnya dilakukan secara individu oleh penulis sendiri.
    
    \item \textbf{Variabel:} Pada tugas akhir ini, ditetapkan variabel bebas "perangkat lunak pembantu (\textit{helper}) yang dikembangkan" serta variabel terikat "keberhasilan penampilan notifikasi secara \textit{native} di Windows dari perangkat lunak yang berjalan di dalam WSL" dan "keberhasilan pengontrolan media secara \textit{native} di Windows dari perangkat lunak pemutar media yang berjalan di dalam WSL".
    
    \item \textbf{Hipotesis:} "Pengembangan perangkat lunak yang tepat sebagai pelengkap Windows Subsystem for Linux dapat memungkinkan notifikasi-notifikasi dari aplikasi Linux tertampil dengan baik di \textit{shell} Windows serta memungkinkan pengontrolan media di aplikasi Linux secara langsung melalui \textit{shell} Windows, sehingga meningkatkan usabilitas perangkat-perangkat lunak Linux yang berjalan di dalam Windows Subsystem for Linux dan meningkatkan pengalaman pengguna (\textit{user experience}) dalam penggunaan Windows Subsystem for Linux."
        
    \item \textbf{Keterbatasan penelitian:} Pengembangan fungsionalitas Windows Subsystem for Linux (WSL) hanya dibatasi pada fungsionalitas-fungsionalitas yang memanfaatkan D-Bus. Lebih spesifiknya, pengembangan fungsionalitas Windows Subsystem for Linux dalam tugas akhir ini hanya terbatas pada sistem penanganan notifikasi dan sistem kontrol media (MPRIS).
\end{enumerate}


\section{Manfaat Penelitian}

\begin{enumerate}
    \item Mengetahui mekanisme mendetail tentang cara kerja protokol dan sistem D-Bus di sistem operasi Linux serta perannya dalam memungkinkan fungsi penanganan notifikasi dan kontrol media.

    \item Meningkatkan cakupan integrasi Windows Subsystem for Linux (WSL) dengan Windows khususnya pada aspek penanganan notifikasi dan kontrol media dengan bantuan perangkat lunak FancyWSL yang telah dikembangkan.

    \item Membuat hasil pengerjaan dalam tugas akhir ini tersedia secara publik di \textit{platform} GitHub agar orang-orang lain dapat ikut memahami, meneliti, dan bahkan melakukan peningkatan pada perangkat lunak yang bersangkutan.
\end{enumerate}


\section{Sistematika Penulisan}

\noindent Bab I berisi pendahuluan yang terdiri dari latar belakang, perumusan masalah, tujuan penelitian, batasan penelitian, manfaat penelitian, serta sistematika penulisan tugas akhir ini.

\noindent Bab II membahas tinjauan pustaka dan dasar teori yang berkaitan dengan teknologi-teknologi yang digunakan pada topik utama tugas akhir ini.

\noindent Bab III membahas metodologi penelitian yang terdiri dari alat dan bahan tugas akhir, alur tugas akhir, metode yang digunakan, serta metode analisis data.

\noindent Bab IV berisi hasil dan pembahasan inti tugas akhir ini.

\noindent Bab V membahas kesimpulan seluruh pengerjaan tugas akhir ini serta saran dan masukan yang dapat meningkatkan tugas akhir ini lebih lanjut.
