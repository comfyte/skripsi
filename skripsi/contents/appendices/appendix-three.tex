\chapter*{LAMPIRAN}

\section{Hasil Pengumpulan Data Pengujian Perangkat Lunak FancyWSL}

\subsection{\textit{Time to Task}}

\autoref{time-to-task-data} berikut menjabarkan data komplet yang telah dikumpulkan mengenai \textit{time to task}.

\begin{table}[h]
    \centering
    \caption{Data \textit{time to task} pada seluruh partisipan}
    \label{time-to-task-data}
    \begin{tabularx}{\linewidth}{|c|X|X|X|X|X|} \hline
        \multirow{2}{*}{\textbf{No.}} & \multirow{2}{*}{\textbf{Inisial Nama}} & \multicolumn{2}{c|}{\textbf{Tanpa FancyWSL}} & \multicolumn{2}{c|}{\textbf{Dengan FancyWSL}}\\ \cline{3-6}
        & & \textbf{Atensi terhadap Judul Lagu} & \textbf{Interaksi Penjedaan Media} & \textbf{Atensi terhadap Judul Lagu} & \textbf{Interaksi Penjedaan Media}\\ \hline
        1 & H. A. H & 7,36 detik & 4,11 detik & 2,45 detik & 6,76 detik\\ \hline
        2 & K. M. & 16,71 detik & 11,08 detik & 2,96 detik & 7,28 detik\\ \hline
    \end{tabularx}
\end{table}

\subsection{\textit{System Usability Scale}}

\autoref{system-usability-scale-data} berikut menjabarkan data komplet yang telah dikumpulkan mengenai \textit{system usability scale} (SUS).

\begin{table}[h]
    \centering
    \caption{Data \textit{time to task} pada seluruh partisipan}
    \label{system-usability-scale-data}
    \begin{tabularx}{\linewidth}{|c|p{3cm}|X|X|X|X|X|X|X|X|X|X|} \hline
        \multirow{2}{*}{\textbf{No.}} & \multirow{2}{*}{\textbf{Inisial Nama}} & \multicolumn{10}{c|}{\textbf{Jawaban Pertanyaan-Pertanyaan \textit{System Usability Scale}}}\\ \cline{3-12}
        & & \textbf{1} & \textbf{2} & \textbf{3} & \textbf{4} & \textbf{5} & \textbf{6} & \textbf{7} & \textbf{8} & \textbf{9} & \textbf{10}\\ \hline
        1 & H. A. H & 4 & 1 & 5 & 1 & 3 & 3 & 5 & 1 & 3 & 1\\ \hline
        2 & K. M. & 4 & 1 & 5 & 1 & 4 & 3 & 5 & 2 & 3 & 2\\ \hline
    \end{tabularx}
\end{table}
