\chapter{Pendahuluan}

\section{Latar Belakang}

% Paragraph Topic: Various OS used by home computers, business/corporate/enterprise computers, server computers, etc.

% Paragraph Topic: The fact that developers tend to use Linux (or Unix-like OSes) over Windows
Penggunaan komputer, terutama dalam bidang pengembangan perangkat lunak, dipenuhi oleh pilihan sistem operasi yang beragam. Dalam survei pengembang tahunan yang diselenggarakan oleh StackOverflow, pada tahun 2022 menunjukkan bahwa secara umum, Windows menempati urutan pertama sistem operasi yang paling banyak digunakan oleh pengembang perangkat lunak, baik dalam konteks penggunaan personal dengan persentase 62,33\% maupun dalam konteks penggunaan profesional dengan persentase 48,82\% \cite{stackoverflow-developer-survey-2022-most-popular-os}. Posisi Windows diikuti oleh sistem operasi Linux yang menempati posisi kedua, dengan 

Windows Subsystem for Linux (WSL) merupakan teknologi yang memungkinkan pengguna menjalankan perangkat lunak dan peralatan-peralatan pengembangan (\textit{development tools}) berbasis Linux secara langsung di dalam sistem operasi Windows 10 atau Windows 11 tanpa memerlukan pergantian sistem operasi. Didasari oleh banyaknya permintaan pengguna untuk lingkungan pengembangan (\textit{development environment}) di Windows yang lebih baik, Windows Subsystem for Linux pertama kali diperkenalkan oleh Microsoft melalui pembaruan Windows 10 bertajuk "Anniversary Update" (pembaruan Juli 2016) sebagai fitur eksperimental (\textit{beta}) dengan nama "Bash on Ubuntu on Windows". Nama awal tersebut kemudian dihilangkan dan diubah menjadi cukup "Windows Subsystem for Linux" untuk menandakan dibukanya dukungan distribusi Linux di luar Ubuntu. Teknologi ini terus berkembang seiring waktu dan umpan balik dari pengguna hingga melepas titel "\textit{beta}"-nya dan menjadi fitur stabil secara resmi pada perilisan pembaruan Windows 10 bertajuk "Fall Creators Update" pada September 2019. Pada tahun 2020, seiring dengan peningkatan versi menjadi bernama Windows Subsystem for Linux versi kedua (WSL2), Windows Subsystem for Linux melalui perubahan arsitektur besar yang meningkatkan kompatibilitas terhadap fungsi-fungsi Linux (\textit{system calls}) secara signifikan dengan dibawanya \textit{kernel} Linux sungguhan yang tervirtualisasi alih-alih hanya mengandalkan lapisan translasi (\textit{translation layer}).
% https://devblogs.microsoft.com/commandline/bash-on-ubuntu-on-windows-download-now-3/
% https://blogs.windows.com/windowsdeveloper/2016/03/30/run-bash-on-ubuntu-on-windows/
% TODO: Cari sitasi/sumber yang pas untuk waktu perilisan WSL pertama di W10 Anniversary Update

Pada perilisan sistem operasi Windows 11, pengembangan Windows Subsystem for Linux memasuki babak baru dengan diresmikannya dukungan penjalanan perangkat lunak berantarmuka grafis (\textit{graphical user interface} atau GUI) melalui fitur bernama "Windows Subsystem for Linux GUI" (WSLg) yang berbasis pada peningkatan yang dilakukan pada WSL2. Hal ini merupakan kemajuan yang signifikan karena kini pengguna tidak lagi terbatas pada perangkat lunak berantarmuka baris perintah (\textit{command line}) saja. Perkembangan ini membuka dukungan yang lebih luas kepada perangkat-perangkat lunak berbasis Linux yang ada, beberapa di antaranya berantarmuka grafis.

Saat ini, Windows Subsystem for Linux telah terintegrasi dengan cukup baik di dalam lingkungan antarmuka Windows secara keseluruhan berkat peningkatan-peningkatan yang telah dilakukan. Pengguna dapat dengan mudah meluncurkan aplikasi atau perangkat lunak berbasis Linux secara langsung di Windows dan perangkat lunak tersebut akan tampil secara harmonis di samping perangkat-perangkat lunak Windows \textit{native} lainnya dan bertindak layaknya perangkat-perangkat lunak \textit{native} tersebut. Sistem perjendelaan (\textit{windowing}) sederhana seperti operasi \textit{maximize}, \textit{restore}, dan \textit{minimize} pada perangkat-perangkat lunak berbasis Linux tersebut bertindak layaknya perangkat lunak Windows \textit{native} pada umumnya sesuai ekspektasi pengguna; dalam sebagian besar kasus penggunaan (\textit{use case}), tingkat integrasi ini sudah cukup dalam membuat perangkat-perangkat lunak berbasis Linux tersebut terasa harmonis. Namun, begitu pengguna mencoba melakukan operasi-operasi yang lebih dari itu, seperti penerimaan notifikasi, \textit{window snapping}, dan kontrol media pada aplikasi-aplikasi musik, pengguna akan mendapati bahwa beberapa operasi tersebut tidak bisa dilakukan karena belum didukung. Dengan kata lain, masih terdapat disparitas pada fungsionalitas sistem perjendelaan antara perangkat-perangkat lunak berbasis Linux yang berjalan di WSL dan perangkat-perangkat lunak berbasis Windows yang berjalan secara \textit{native} dikarenakan belum sempurnanya integrasi Windows Subsystem for Linux GUI ke dalam lingkungan Windows secara keseluruhan.

Melalui skripsi ini, penulis mengeksplor potensial-potensial penyempurnaan yang dapat dilakukan untuk meningkatkan situasi yang ada. Penulis berfokus pada peningkatan dalam dua aspek, yaitu aspek penampilan notifikasi dan aspek kontrol media. Namun, dapat dipahami bahwa implementasi sistem Linux di WSL bukan merupakan implementasi yang sempurna dan terdapat celah-celah fungsionalitas yang umumnya ada pada distribusi Linux normal, mengingat WSL tidak didesain sebagai sebuah distribusi Linux utuh, yang dapat menghambat pengimplementasian kedua aspek awal tersebut, sehingga diperlukan pula pengimplementasian hal-hal penunjang yang mungkin diperlukan. Penulis berharap bahwa penyempurnaan yang penulis lakukan dapat bermanfaat bagi pengguna-pengguna sistem operasi Windows yang bergantung pada Windows Subsystem for Linux. Penulis percaya bahwa usaha yang dilakukan dalam penelitian ini memiliki efek jangka panjang meningkatkan pengalaman pengguna (\textit{user experience)} secara keseluruhan. Di samping itu, tidak menutup kemungkinan bahwa penyempurnaan yang penulis lakukan dapat bermanfaat pula bagi tim pengembang \textit{upstream} yang bekerja mengembangkan fitur Windows Subsystem for Linux ini.

\section{Rumusan Masalah}

"Bagaimana menciptakan lingkup integrasi yang lebih menyeluruh dan lebih komprehensif dibandingkan dengan lingkup integrasi yang telah ada, terutama dalam aspek penampilan notifikasi dan pengontrolan media, guna memaksimalkan kepuasan pengguna dalam menggunakan perangkat-perangkat lunak berbasis Linux dan berantarmuka grafis (GUI) di Windows Subsystem for Linux (WSL)?"


\section{Tujuan Penelitian}

\begin{enumerate}
    \item Memeriksa bahwa benar belum ada implementasi bawaan (\textit{built-in}) fungsionalitas penanganan notifikasi dan kontrol media secara bawaan di Windows Subsystem for Linux versi terbaru.

    \item Memvalidasi adanya permintaan (\textit{demand}) yang cukup dari pengguna terhadap fungsionalitas penanganan notifikasi dan kontrol media di Windows Subsystem for Linux serta, bila ada, mengeksplorasi solusi-solusi sementara dan/atau pihak ketiga yang telah mereka lakukan.

    \item Menyelidiki cara kerja penanganan notifikasi dan kontrol media secara standar di sistem operasi Linux dengan menggunakan instalasi Linux lain.
    
    \item Menyelidiki dan memperbaiki disparitas yang ada di implementasi Windows Subsystem for Linux dibandingkan dengan instalasi Linux sungguhan yang berpotensi menghalangi langkah-langkah pengimplementasian pemrosesan notifikasi dan kontrol media selanjutnya.

    \item Mengembangkan perangkat lunak di sisi Windows Subsystem for Linux (WSL) yang dapat memproses notifikasi dan kontrol media untuk kemudian dihubungkan dengan perangkat lunak di sisi Windows.

    \item Mengeksplor dan memilih metode yang tepat untuk menghubungkan sisi Windows Subsystem for Linux (WSL) dengan sisi Windows agar perangkat lunak di kedua sisi dapat saling berkomunikasi.

    \item Mengembangkan perangkat lunak di sisi Windows yang bertindak sebagai penjembatan (\textit{bridge}) yang akan mengonsumsi informasi yang diberikan oleh perangkat lunak dari sisi Windows Subsystem for Linux (WSL).

    \item Melakukan pengujian untuk memastikan bahwa seluruh pengembangan yang telah dilakukan dapat bekerja dengan baik dan harmonis.
\end{enumerate}

%-------------------------------------------------	

\section{Batasan Penelitian}

\begin{enumerate}
    \item \textbf{Objek penelitian:} Penelitian dilakukan pada instalasi perangkat lunak Windows Subsystem for Linux versi 2.0.9 di sebuah perangkat komputer bersistem operasi Microsoft Windows 11 Pro berarsitektur x86\_64 (64-bit).
    
    \item \textbf{Metode penelitian:} 
    
    \item \textbf{Waktu dan tempat penelitian:} Penelitian dan percobaan dilakukan dari April 2023 hingga Desember 2023 di perangkat komputer beridentitas "COM25" di Laboratorium Jaringan Komputer dan Aplikasi Terdistribusi, Departemen Teknik Elektro dan Teknologi Informasi, Universitas Gadjah Mada. Untuk memudahkan pengerjaan, dimungkinkan pengaksesan perangkat komputer secara jarak jauh (\textit{remote access}) menggunakan bantuan perangkat lunak Microsoft Remote Desktop.
    
    \item \textbf{Populasi dan sampel:} Pengerjaan dan evaluasi seluruhnya dilakukan secara individu oleh penulis sendiri.
    
    \item \textbf{Variabel:} 
    
    \item \textbf{Hipotesis:} Pengembangan perangkat lunak yang tepat sebagai pelengkap Windows Subsystem for Linux dapat memungkinkan notifikasi-notifikasi dari aplikasi Linux tertampil dengan baik di \textit{shell} Windows serta memungkinkan pengontrolan media di aplikasi Linux secara langsung melalui \textit{shell} Windows, sehingga meningkatkan usabilitas perangkat-perangkat lunak Linux yang berjalan di dalam Windows Subsystem for Linux dan meningkatkan pengalaman pengguna (\textit{user experience}) dalam penggunaan Windows Subsystem for Linux.

    % \begin{enumerate}
    %     \item Implementasi Windows Subsystem for Linux yang telah ada belum memiliki fungsionalitas penyampaian notifikasi dan pengontrolan media.
    %     \item Pengimplementasian sistem penyampaian notifikasi dan pengontrolan media pada implementasi WSL meningkatkan usabilitas perangkat-perangkat lunak yang dapat digunakan di lingkungan WSL, sehingga meningkatkan kepuasan dan pengalaman pengguna.
    % \end{enumerate}
    
    % Pengimplementasian fitur penampilan notifikasi dan pengontrolan media yang sebelumnya belum ada akan meningkatkan kepuasan pengguna secara keseluruhan dalam menggunakan Windows Subsystem for Linux.
    
    \item \textbf{Keterbatasan penelitian:} Pengembangan fungsionalitas Windows Subsystem for Linux hanya dibatasi pada sistem penyampaian notifikasi dan sistem pengontrolan media.
    
    % Penelitian dan percobaan hanya dilakukan pada sistem operasi Windows 11 berarsitektur x86\_64, Windows Subsystem for Linux versi 2, dan distribusi Ubuntu yang berjalan secara tunggal tanpa adanya distribusi lain.
\end{enumerate}

\section{Manfaat Penelitian}

Manfaat penelitian didefinisikan sebagai manfaat yang diperoleh apabila 
skripsi telah selesai dilakukan. Manfaat skripsi pada umumnya berbentuk daftar. 
Manfaat penelitian dapat berupa manfaat bagi dunia akademik dan atau masyarakat.

\section{Sistematika Penulisan}

Sistematika penulisan berisi pembahasan apa yang akan ditulis di setiap bab. 
Sistematika pada umumnya berupa paragraf yang setiap paragraf mencerminkan 
bahasan setiap Bab. Contoh:

\noindent Bab I membahas tentang pendahuluan yang berisi latar belakang, perumusan masalah 
dan tujuan penelitian. 

\noindent Bab II berisi tentang metodologi penelitian yang terdiri dari desain penelitian, sumber data, Teknik pengumpulan data dan Teknik analisis data.

\noindent Dan seterusnya.

