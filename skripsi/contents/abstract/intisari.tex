Windows Subsystem for Linux (WSL) merupakan solusi yang ditawarkan oleh Microsoft untuk menjalankan perangkat lunak, aplikasi, ataupun peralatan (\textit{tools}) Linux secara langsung di dalam lingkungan sistem operasi Windows. WSL diperkenalkan tidak beberapa lama setelah perilisan Windows 10; sejak perilisan Windows 11, WSL memiliki kemampuan baru, yaitu penampilan perangkat lunak Linux secara grafis (\textit{graphical user interface}). Namun, tingkat pengintegrasian WSL dengan Windows masih belum sempurna dan terdapat sejumlah aspek yang perlu diimplementasikan. Tugas akhir ini membahas pengimplementasian dua aspek yang melibatkan komunikasi dengan bus perpesanan D-Bus, sistem penanganan notifikasi dan sistem kontrol media, yang dapat meningkatkan aspek pengalaman pengguna (\textit{user experience}).

\noindent{Kata kunci} : WSL, Linux, D-Bus, notifikasi, kontrol media
