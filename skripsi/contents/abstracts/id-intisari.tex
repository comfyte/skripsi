Windows Subsystem for Linux (WSL) merupakan solusi yang ditawarkan oleh Microsoft untuk menjalankan perangkat lunak, aplikasi, ataupun peralatan (\textit{tools}) Linux secara langsung di dalam lingkungan sistem operasi Windows. WSL diperkenalkan tidak beberapa lama setelah perilisan Windows 10; sejak perilisan Windows 11, WSL memiliki kemampuan baru, yaitu penampilan perangkat lunak Linux secara grafis (\textit{graphical user interface}). Namun, tingkat pengintegrasian WSL dengan Windows masih belum sempurna dan terdapat sejumlah aspek yang perlu diimplementasikan.

Tugas akhir ini membahas pengimplementasian dua aspek yang melibatkan komunikasi dengan bus perpesanan D-Bus, sistem penanganan notifikasi dan sistem kontrol media, yang dapat meningkatkan aspek pengalaman pengguna (\textit{user experience}). Pengimplementasian dilakukan melalui sebuah perangkat lunak yang dinamakan "FancyWSL" yang berupa perangkat lunak \textit{daemon} yang berjalan di latar belakang dan bertugas menyediakan fasilitas penanganan notifikasi dan pengontrolan media bagi perangkat-perangkat lunak yang berjalan di dalam WSL.

Tugas akhir ini membuktikan bahwa pengimplementasian kedua aspek berikut mungkin dilakukan berkat sistem Linux di balik WSL yang bersifat fleksibel dan modular. Selain itu, sejumlah pengujian yang dilakukan pada tugas akhir ini seperti pengujian berdasarkan \textit{system usability scale} (SUS) dan pengujian \textit{time to task} membuktikan bahwa keberadaan perangkat lunak FancyWSL meningkatkan usabilitas perangkat-perangkat lunak yang berjalan di dalam WSL, terutama perangkat-perangkat lunak yang mengandalkan penyampaian notifikasi dan/atau pengontrolan media sebagai salah satu bagian interaksinya.

\noindent{Kata kunci} : WSL, Linux, D-Bus, notifikasi, kontrol media
