\chapter*{LAMPIRAN}

\section{Hasil Pengumpulan Data Pengujian Perangkat Lunak FancyWSL}

\subsection{Informasi Umum}

\begin{enumerate}
    \item \textbf{Partisipan kesatu}
    \begin{itemize}
        \item \textbf{Waktu partisipasi:} 2024/01/18 1:35:57 AM GMT+7
        \item \textbf{Jenis kelamin:} Pria
        \item \textbf{Klaster fakultas:} Saintek
        \item \textbf{Domisili:} Sleman, Yogyakarta
        \item \textbf{Saran/kritik/masukan khusus untuk FancyWSL:} Ketika menekan tombol pause di WSL, Spotify tidak terpause, ke depannya bisa diperbaiki fungsionalitas tersebut, dan nama di FancyWSL bisa diganti karena di Windows masih memiliki nama "Python 3.11".
    \end{itemize}

    \item \textbf{Partisipan kedua}
    \begin{itemize}
        \item \textbf{Waktu partisipasi:} 2024/01/18 1:59:41 PM GMT+7
        \item \textbf{Jenis kelamin:} Pria
        \item \textbf{Klaster fakultas:} Saintek
        \item \textbf{Domisili:} Semarang
        \item \textbf{Saran/kritik/masukan khusus untuk FancyWSL:} (dikosongkan)
    \end{itemize}

    \item \textbf{Partisipan ketiga}
    \begin{itemize}
        \item \textbf{Waktu partisipasi:} 2024/01/22 9:22:40 PM GMT+7
        \item \textbf{Jenis kelamin:} Pria
        \item \textbf{Klaster fakultas:} Saintek
        \item \textbf{Domisili:} Jakarta Utara
        \item \textbf{Saran/kritik/masukan khusus untuk FancyWSL:} Tidak ada sejauh ini

    \end{itemize}

    \item \textbf{Partisipan keempat}
    \begin{itemize}
        \item \textbf{Waktu partisipasi:} 2024/01/22 9:53:14 PM GMT+7
        \item \textbf{Jenis kelamin:} Pria
        \item \textbf{Klaster fakultas:} Saintek
        \item \textbf{Domisili:} Depok
        \item \textbf{Saran/kritik/masukan khusus untuk FancyWSL:} notifikasi jangan menggunakan nama "python 3.11"
    \end{itemize}

    \item \textbf{Partisipan kelima}
    \begin{itemize}
        \item \textbf{Waktu partisipasi:} 2024/01/22 10:21:19 PM GMT+7
        \item \textbf{Jenis kelamin:} Pria
        \item \textbf{Klaster fakultas:} Saintek
        \item \textbf{Domisili:} Daerah Istimewa Yogyakarta
        \item \textbf{Saran/kritik/masukan khusus untuk FancyWSL:} Sudah sangat bagus dan akan lebih mudah digunakan lagi apabila FancyWSL dapat berjalan langsung saat start-up. Jadi hanya perlu menjalankan installer sekali lalu user tidak perlu menjalankannya kembali.
    \end{itemize}

    \item \textbf{Partisipan keenam}
    \begin{itemize}
        \item \textbf{Waktu partisipasi:} 2024/01/22 10:37:43 PM GMT+7
        \item \textbf{Jenis kelamin:} Pria
        \item \textbf{Klaster fakultas:} Saintek
        \item \textbf{Domisili:} Yogyakarta
        \item \textbf{Saran/kritik/masukan khusus untuk FancyWSL:} sudah bagus
    \end{itemize}
\end{enumerate}

\subsection{\textit{Time to Task}}

\autoref{time-to-task-data} berikut menjabarkan data komplet yang telah dikumpulkan mengenai \textit{time to task}.

\begin{table}[h]
    \centering
    \caption{Data \textit{time to task} pada seluruh partisipan}
    \label{time-to-task-data}
    \begin{tabularx}{\linewidth}{|c|X|X|X|X|} \hline
        \multirow{2}{*}{\textbf{Partisipan}} & \multicolumn{2}{c|}{\textbf{Tanpa FancyWSL}} & \multicolumn{2}{c|}{\textbf{Dengan FancyWSL}}\\ \cline{2-5}
        & \textbf{Atensi terhadap Judul Lagu} & \textbf{Interaksi Penjedaan Media} & \textbf{Atensi terhadap Judul Lagu} & \textbf{Interaksi Penjedaan Media}\\ \hline
        \textbf{1} & 7,36 detik & 4,11 detik & 2,45 detik & 6,76 detik\\ \hline
        \textbf{2} & 16,71 detik & 11,08 detik & 2,96 detik & 7,28 detik\\ \hline
        \textbf{3} & 10,7 detik & 11,35 detik & 2,51 detik & 7,48 detik \\ \hline
        \textbf{4} & 7,68 detik & 5,12 detik & 2,1 detik & 9,69 detik \\ \hline
        \textbf{5} & 6,01 detik & 8,06 detik & 4,7 detik & 8,78 detik \\ \hline
        \textbf{6} & 5 detik & 5,43 detik & 1,75 detik & 5,20 detik \\ \hline
    \end{tabularx}
\end{table}

\subsection{\textit{System Usability Scale}}

\autoref{system-usability-scale-data} berikut menjabarkan data komplet yang telah dikumpulkan mengenai \textit{system usability scale} (SUS).

\begin{table}[h]
    \centering
    \caption{Data \textit{time to task} pada seluruh partisipan}
    \label{system-usability-scale-data}
    \begin{tabularx}{\linewidth}{|c|X|X|X|X|X|X|X|X|X|X|} \hline
        \multirow{2}{*}{\textbf{Partisipan}} & \multicolumn{10}{c|}{\textbf{Jawaban Pertanyaan-Pertanyaan \textit{System Usability Scale}}}\\ \cline{2-11}
        & \textbf{1} & \textbf{2} & \textbf{3} & \textbf{4} & \textbf{5} & \textbf{6} & \textbf{7} & \textbf{8} & \textbf{9} & \textbf{10}\\ \hline
        \textbf{1} & 4 & 1 & 5 & 1 & 3 & 3 & 5 & 1 & 3 & 1\\ \hline
        \textbf{2} & 4 & 1 & 5 & 1 & 4 & 3 & 5 & 2 & 3 & 2\\ \hline
        \textbf{3} & 5 & 1 & 5 & 2 & 5 & 1 & 5 & 1 & 1 & 1\\ \hline
        \textbf{4} & 4 & 1 & 4 & 3 & 4 & 4 & 3 & 2 & 2 & 4\\ \hline
        \textbf{5} & 5 & 2 & 4 & 1 & 5 & 2 & 5 & 1 & 1 & 1\\ \hline
        \textbf{6} & 5 & 1 & 5 & 1 & 5 & 1 & 5 & 1 & 1 & 1\\ \hline
    \end{tabularx}
\end{table}
