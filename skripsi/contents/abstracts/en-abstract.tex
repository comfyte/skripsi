\textit{
Windows Subsystem for Linux (WSL) was released by Microsoft to support running Linux applications and software directly within the Windows operating system. WSL was introduced shortly after the introduction of Windows 10; since Windows 11, WSL gained a new support for running graphical applications directly within the Windows user interface. The implementation of these still lacks several aspects, such as notification handling support and media control support.}

\textit{This thesis discusses about these two aspects, the background, the detailed implementation, and the final testing. The implementation is done in the form of a daemon-style helper software called "FancyWSL" that uses the Python programming language. This software provides interfaces for Linux softwares running on WSL in order to be able to send notifications and leverage the universal media control feature of Windows.}

\textit{At the end of this research, after some thorough tests based on system usability scale (SUS) and time-to-task, the collected data suggests that the introduction of this "FancyWSL" helper software does indeed improve the quality of interaction between users and Linux softwares running on WSL.}

\noindent\textbf{Keywords} : WSL, Linux, D-Bus, notification, media control
