Intisari ditulis menggunakan bahasa Indonesia dengan jarak antar baris 1 spasi dan maksimal 1 halaman. Intisari sekurang-kurangnya berisi tentang latar belakang dan tujuan penelitian, metodologi yang digunakan, hasil penelitian, kesimpulan dan implikasi, dan Kata kunci yang berhubungan dengan penelitian.

Kata Kunci ditulis maksimal 5 kata yang paling berhubungan dengan isi skripsi. Silakan mengacu pada ACM / IEEE \textit{Computing classification} jika Anda adalah mahasiswa Sarjana TI \textcolor{blue}{http://www.acm.org/about/class/} atau mengacu kepada IEEE keywords \textcolor{blue}{http://www.ieee.org/documents/taxonomy\_v101.pdf} jika Anda berasal dari Prodi Sarjana TE.

\noindent{Kata kunci} : Kata kunci 1, Kata kunci 2, Kata kunci 3, Kata kunci 4, Kata kunci 5

\vspace{1cm}

%HAPUS YANG TIDAK PERLU
%-------------------------------------------------
\noindent\fbox{%
	\parbox{\textwidth}{%
\textbf{Contoh Abstrak Teknik Elektro:} \\

\hspace{1cm} "Penelitian ini bertujuan untuk mengembangkan sistem pengendalian suhu ruangan dengan menggunakan microcontroller. Metodologi yang digunakan adalah desain sirkuit, implementasi sistem pengendalian, dan pengujian performa. Hasil penelitian menunjukkan 
bahwa sistem pengendalian suhu ruangan yang dikembangkan mampu mengendalikan suhu ruangan dengan akurasi sebesar ±0,5°C. Kesimpulan dari penelitian ini adalah sistem pengendalian suhu ruangan yang dikembangkan efektif dan efisien. \\

Kata kunci: microcontroller, sistem pengendalian suhu, akurasi."
\vspace{5mm}

\textbf{Contoh Abstrak Teknik Biomedis:} \\

\hspace{1cm} "Penelitian ini bertujuan untuk mengevaluasi keefektifan prototipe alat pemantau denyut jantung berbasis elektrokardiogram (ECG) untuk pasien jantung. Metodologi yang digunakan meliputi desain dan pembuatan prototipe, pengujian dengan pasien, dan analisis data. Hasil penelitian menunjukkan bahwa prototipe alat pemantau denyut jantung berbasis ECG memiliki 
akurasi yang baik dan mampu memantau denyut jantung pasien secara efektif. Kesimpulan dari penelitian ini adalah prototipe alat pemantau denyut jantung berbasis ECG merupakan solusi 
yang efektif dan efisien untuk memantau pasien jantung. \\

Kata kunci: elektrokardiogram, alat pemantau denyut jantung, akurasi."
\vspace{5mm}

	}%
}

%-------------------------------------------------

\noindent\fbox{%
	\parbox{\textwidth}{%
		\textbf{Contoh Abstrak Teknologi Informasi:} \\
		
\hspace{1cm} "Penelitian ini bertujuan untuk mengevaluasi keamanan dan privasi pengguna aplikasi media sosial terpopuler. Metodologi yang digunakan meliputi analisis kebijakan privasi dan pengaturan keamanan, pengujian penetrasi, dan survei pengguna. Hasil penelitian 
menunjukkan bahwa beberapa aplikasi media sosial memiliki kebijakan privasi yang kurang jelas dan rendahnya tingkat keamanan. Kesimpulan dari penelitian ini adalah pentingnya meningkatkan kebijakan privasi dan tingkat keamanan pada aplikasi media sosial untuk melindungi privasi dan data pengguna. \\
		
Kata kunci: media sosial, keamanan, privasi, pengguna."
\vspace{5mm}
		
	}%
}